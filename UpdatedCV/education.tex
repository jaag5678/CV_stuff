%----------------------------------------------------------------------------------------
%	EDUCATION SECTION
%----------------------------------------------------------------------------------------

\section{Education}

    \cventry
        {2021-Today}
        {PhD Computer Science}
        {McGill University}
        {Montreal}
        {}
        {
        {
            Courses Taken
            \begin{multicols}{2}
                \begin{enumerate}
                  \item High Level Synthesis of Digital Systems - Winter 2022.
                  \item Mathematical Tools for Computer Science - Fall 2022.
                  \item Advanced Computer Systems - Fall 2023.
                \end{enumerate}
            \end{multicols}
              }
    }  % Arguments not required can be left empty

    \cventry
        {2018--2021}
        {Masters of Computer Science - Thesis}
        {McGill University}
        {Montreal}
        {}
        {
            Courses taken
            \begin{multicols}{2}
                \begin{enumerate}
                  \item Machine Learning - Fall 2018.
                  \item Teaching in Computer Science - Fall 2018.
                  \item Compiler Design - Winter 2019.
                  \item Epistemic \& Sociocultural Understanding of Computer Science - Winter 2019. 
                  \item Semantics of Programming Languages - Seminar course at University of Montreal - Fall 2019.
                  \item Meta-Programming - Winter 2020.
                \end{enumerate}
            \end{multicols}      
        }  % Arguments not required can be left empty
    
    \cventry
        {2014--2018}
        {Bachelor of Information Technology}
        {College of Engineering}
        {Pune}
        {}
        {
            \textbf{Relevant} Courses Taken 
            \begin{multicols}{2}
                \begin{enumerate}
                    \item Data Structures and Algorithms - 2015.
                    \item Digital Systems - 2015.
                    \item Discrete Structures and Graph Theory - 2015.
                    \item Microprocessor Techniques - 2016.
                    \item Principles of Programming Languages - 2016.
                    \item Theory of Computer Science -2016.
                    \item Operating Systems - 2016.
                    \item Algorithms and Complexity - 2017.
                    \item Assemblers and Compilers - 2017.
                \end{enumerate}
            \end{multicols}
        }

\section{Masters Thesis}

    \cvitem{Title}{\textbf{Analysis of the ECMAscript memory model: a program transformation perspective}}
    \cvitem{Supervisors}{\textbf{Professor Clark Verbrugge}}
    \cvitem{Abstract}{Concurrent programs have been shown to give us tremendous performance benefits compared to their sequential counterparts. 
    With the addition of several hardware features such as read/write buffers, speculation, etc., more efficient forms of concurrent memory accesses are introduced. 
    Known as relaxed memory accesses, they are used to gain substantial improvement in the performance of concurrent programs.
    A relaxed memory consistency model specifically describes the semantics of such accesses for a particular programming language. 
    Historically, such semantics are often ill-defined or misunderstood, and have been shown to conflict with common program transformations essential for the performance of programs. 
    In this thesis, we give a formal declarative (axiomatic) style description of the ECMAScript relaxed memory consistency model. 
    We analyze the impact of this model on two common program transformations, viz. instruction reordering and elimination. 
    We give a conservative proof under which such an optimization is allowed for relaxed memory accesses. 
    We use this result to reason about the validity of reordering accesses outside loops under the same model. 
    We conclude this thesis by eliciting the limitations of our approach, critique on the semantics of the model, possible future work using our results, and pending foundational questions that we discovered while working on this thesis.}
    \cvitem{Link}{\url{https://escholarship.mcgill.ca/concern/theses/7p88cn613?locale=en}}